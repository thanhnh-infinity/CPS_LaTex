%%%%%%%%%%%%%%%%%%%%%%%%%%%%%%%%%%%%%%%%%
% Beamer Presentation
% LaTeX Template
% Version 1.0 (10/11/12)
%
% This template has been downloaded from:
% http://www.LaTeXTemplates.com
%
% License:
% CC BY-NC-SA 3.0 (http://creativecommons.org/licenses/by-nc-sa/3.0/)
%
%%%%%%%%%%%%%%%%%%%%%%%%%%%%%%%%%%%%%%%%%

%----------------------------------------------------------------------------------------
%	PACKAGES AND THEMES
%----------------------------------------------------------------------------------------

\documentclass{beamer}

\mode<presentation> {

% The Beamer class comes with a number of default slide themes
% which change the colors and layouts of slides. Below this is a list
% of all the themes, uncomment each in turn to see what they look like.

%\usetheme{default}
%\usetheme{AnnArbor}
%\usetheme{Antibes}
%\usetheme{Bergen}
%\usetheme{Berkeley}
%\usetheme{Berlin}
%\usetheme{Boadilla}
%\usetheme{CambridgeUS}
%\usetheme{Copenhagen}
%\usetheme{Darmstadt}
%\usetheme{Dresden}
%\usetheme{Frankfurt}
%\usetheme{Goettingen}
%\usetheme{Hannover}
%\usetheme{Ilmenau}
%\usetheme{JuanLesPins}
%\usetheme{Luebeck}
\usetheme{Madrid}
%\usetheme{Malmoe}
%\usetheme{Marburg}
%\usetheme{Montpellier}
%\usetheme{PaloAlto}
%\usetheme{Pittsburgh}
%\usetheme{Rochester}
%\usetheme{Singapore}
%\usetheme{Szeged}
%\usetheme{Warsaw}

% As well as themes, the Beamer class has a number of color themes
% for any slide theme. Uncomment each of these in turn to see how it
% changes the colors of your current slide theme.

%\usecolortheme{albatross}
%\usecolortheme{beaver}
%\usecolortheme{beetle}
%\usecolortheme{crane}
%\usecolortheme{dolphin}
%\usecolortheme{dove}
%\usecolortheme{fly}
%\usecolortheme{lily}
%\usecolortheme{orchid}
%\usecolortheme{rose}
%\usecolortheme{seagull}
%\usecolortheme{seahorse}
%\usecolortheme{whale}
%\usecolortheme{wolverine}

%\setbeamertemplate{footline} % To remove the footer line in all slides uncomment this line
%\setbeamertemplate{footline}[page number] % To replace the footer line in all slides with a simple slide count uncomment this line

%\setbeamertemplate{navigation symbols}{} % To remove the navigation symbols from the bottom of all slides uncomment this line
}

\usepackage{graphicx} % Allows including images
\usepackage{booktabs} % Allows the use of \toprule, \midrule and \bottomrule in tables
\usepackage{algorithm}
\usepackage{algpseudocode}
\usepackage{amsmath}

\usepackage{listings}
\lstdefinelanguage{clingo}{
  keywordstyle=[1]\usefont{OT1}{cmtt}{m}{n},%
  keywordstyle=[2]\textbf,%
  keywordstyle=[3]\usefont{OT1}{cmtt}{m}{n},%\textit
  alsoletter={\#,\&},%
  keywords=[1]{not,from,import,exists,if,else,return,while,break,and,or,for,in,del,and,class,subclass,operation,has_input,has_output,input_spec,output_spec,init,occ,map,score,failed,succ,type},%
  keywords=[2]{\#const,\#show,\#minimize,\#base,\#theory,\#count,\#external,\#program,\#script,\#end,\#heuristic,\#edge,\#project,\#show},%
  keywords=[3]{&,&dom,&sum,&diff,&show,&minimize},%
  morecomment=[l]{\#\ },%
  morecomment=[l]{\%\ },%
  commentstyle={\color{darkgray}}%
}
\lstset{numberblanklines=false,basicstyle=\ttfamily\small,language=clingo}
\lstset{escapeinside={*@}{@*}}

%----------------------------------------------------------------------------------------
%	TITLE PAGE
%----------------------------------------------------------------------------------------

\title[Truthworthiness value]{Probabilistic Mitigation Strategies} % The short title appears at the bottom of every slide, the full title is only on the title page

\author{Thanh H. Nguyen} % Your name
\institute[NMSU] % Your institution as it will appear on the bottom of every slide, may be shorthand to save space
{
New Mexico State University \\ % Your institution for the title page
\medskip
\textit{tnguyen@cs.nmsu.edu} % Your email address
}
\date{\today} % Date, can be changed to a custom date

\begin{document}

\begin{frame}
\titlepage % Print the title page as the first slide
\end{frame}

% Slide 1:

%\begin{frame}
%\frametitle{Overview} % Table of contents slide, comment this block out to remove it
%\tableofcontents % Throughout your presentation, if you choose to use \section{} and \subsection{} commands, these will automatically be printed on this slide as an overview of your presentation
%\end{frame}

%----------------------------------------------------------------------------------------
%	PRESENTATION SLIDES
%----------------------------------------------------------------------------------------

%------------------------------------------------
\section{Introduction} 

\begin{frame}
	\frametitle{Physical CPS System}
	\begin{definition}
		\label{def:physical_CPS_system} 
		A physical CPS system $S$ is a tuple ($C, A_p, F, R, Prob$) where:
		%
		\begin{list}{$\bullet$}{\itemsep=0pt \parsep=1pt \topsep=1pt \leftmargin=12pt} 
			\item $C$ is a set of physical components.
			\item $A_p$ is a set of tuples $(a, prob_a)$, where $a$ is an action that can be execute over CPS system, and $prob_a$ is the probability of success of action $a$. ($0 \leq prob \leq 100$) or $prob_a = None$ if the probability of success of action a is unknown.
			\item $F$ is a finite set of fluent literals.
			\item $R$ is a set of relations that map each physical component $c \in C$ with a set of physical component properties that are defined in CPS Ontology.  \\
			For any $r \in R$, $r : C \longrightarrow 2^{P}$. $P$ is set of all properties that are defined in CPS ontology. 
			%\item $Pr$ is a set of probabilities of success of actions. Given an action $a \in A$, there is a probability of success of action $a$ denoted by $prob(a) \in Prob$. The value of $prob(a)$ can be $None$ if the probability of success of $a$ is unknown. $Prob = \{prob(a) | a \in A\}$			 
		\end{list}
		%
	\end{definition}
\end{frame}

\begin{frame}[fragile]
	\frametitle{Representation the System}
	\begin{itemize}
		\item {\bf Step 1}: Representation the probability of success of action. The fluent {\tt prob\_success(a,$prob_a$)} denotes that an action $a$ has probability $prob_a$ ($0 \leq prob_a \leq 100$).
		\item {\bf Step 2}: The fluent {\tt prob\_of\_state(prob)} models the propagation by the model to the successor state. The statement {\tt holds(prob\_of\_state(prob),S)} means that at step $S$ of the CPS evolution, the probability of the current state described by this fluent is $prob$ ($0 \leq prob \leq 100$). The initial value at time step 0 is {\tt holds(prob\_of\_state(100),0)} or {\tt prob\_of\_state(0)} = 100.
		\item {\bf Step 3}: Assuming that at step $S$ of evolution, an action $a$ can be executed. The predicate {\tt do(a,S)} denotes that action $a$ is executed at step $S$.
	\end{itemize}
\end{frame}

\begin{frame}[fragile]
	\frametitle{Compute the Probability of success of mitigation strategies}
	\begin{itemize}
		\item {\bf Step 4}: (1) Given the probability of success of mitigation strategies in CPS System at step $S$ : {\tt prob\_of\_state(S)}. (2) At step $S$, an action $a$ is executed ({\tt do(a,S)} holds) and the probability of success of $a$ is {\tt prob\_success(a)}. So the probability of success of CPS system at step $S + 1$ is:
		$$
			prob\_of\_state(S+1) = 
			\begin{cases}
			\dfrac{prob\_of\_state(S)*prob\_success(a)}{100}, \\ \text{ if } prob\_success(a) \neq None \\
			prob\_of\_state(S), \text{ if } prob\_success(a) = None 
			\end{cases} 
		$$
		\item {\bf Step 5}: Finally, assume that $S_{last}$ is the last step of system evolution, the value of {\tt prob\_of\_state($S_{last}$)} represents the probability of success of mitigation strategy $\alpha = a_0...a_{S_{last}}$
	
	\end{itemize}
\end{frame}
%------------------------------------------------

%----------------------------------------------------------------------------------------
%\bibliographystyle{aaai}

%\begin{thebibliography}{}

%\bibitem[\protect\citeauthoryear{Berners-Lee, Hendler, and
%                Lassila}{2001}]{lee01a}
%        Berners-Lee, T.; Hendler, J.; and Lassila, O. 2001. {The Semantics Web.} {\em Scientific American} 284(5):34--43.
        
%\end{thebibliography}
\end{document} 