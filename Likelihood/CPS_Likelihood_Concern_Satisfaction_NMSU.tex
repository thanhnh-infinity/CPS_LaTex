%%%%%%%%%%%%%%%%%%%%%%%%%%%%%%%%%%%%%%%%%
% Beamer Presentation
% LaTeX Template
% Version 1.0 (10/11/12)
%
% This template has been downloaded from:
% http://www.LaTeXTemplates.com
%
% License:
% CC BY-NC-SA 3.0 (http://creativecommons.org/licenses/by-nc-sa/3.0/)
%
%%%%%%%%%%%%%%%%%%%%%%%%%%%%%%%%%%%%%%%%%

%----------------------------------------------------------------------------------------
%	PACKAGES AND THEMES
%----------------------------------------------------------------------------------------

\documentclass{beamer}

\mode<presentation> {

% The Beamer class comes with a number of default slide themes
% which change the colors and layouts of slides. Below this is a list
% of all the themes, uncomment each in turn to see what they look like.

%\usetheme{default}
%\usetheme{AnnArbor}
%\usetheme{Antibes}
%\usetheme{Bergen}
%\usetheme{Berkeley}
%\usetheme{Berlin}
%\usetheme{Boadilla}
%\usetheme{CambridgeUS}
%\usetheme{Copenhagen}
%\usetheme{Darmstadt}
%\usetheme{Dresden}
%\usetheme{Frankfurt}
%\usetheme{Goettingen}
%\usetheme{Hannover}
%\usetheme{Ilmenau}
%\usetheme{JuanLesPins}
%\usetheme{Luebeck}
\usetheme{Madrid}
%\usetheme{Malmoe}
%\usetheme{Marburg}
%\usetheme{Montpellier}
%\usetheme{PaloAlto}
%\usetheme{Pittsburgh}
%\usetheme{Rochester}
%\usetheme{Singapore}
%\usetheme{Szeged}
%\usetheme{Warsaw}

% As well as themes, the Beamer class has a number of color themes
% for any slide theme. Uncomment each of these in turn to see how it
% changes the colors of your current slide theme.

%\usecolortheme{albatross}
%\usecolortheme{beaver}
%\usecolortheme{beetle}
%\usecolortheme{crane}
%\usecolortheme{dolphin}
%\usecolortheme{dove}
%\usecolortheme{fly}
%\usecolortheme{lily}
%\usecolortheme{orchid}
%\usecolortheme{rose}
%\usecolortheme{seagull}
%\usecolortheme{seahorse}
%\usecolortheme{whale}
%\usecolortheme{wolverine}

%\setbeamertemplate{footline} % To remove the footer line in all slides uncomment this line
%\setbeamertemplate{footline}[page number] % To replace the footer line in all slides with a simple slide count uncomment this line

%\setbeamertemplate{navigation symbols}{} % To remove the navigation symbols from the bottom of all slides uncomment this line
}

\usepackage{graphicx} % Allows including images
\usepackage{booktabs} % Allows the use of \toprule, \midrule and \bottomrule in tables
\usepackage{algorithm}
\usepackage{algpseudocode}
\usepackage{amsmath}

\usepackage{listings}
\lstdefinelanguage{clingo}{
  keywordstyle=[1]\usefont{OT1}{cmtt}{m}{n},%
  keywordstyle=[2]\textbf,%
  keywordstyle=[3]\usefont{OT1}{cmtt}{m}{n},%\textit
  alsoletter={\#,\&},%
  keywords=[1]{not,from,import,exists,if,else,return,while,break,and,or,for,in,del,and,class,subclass,operation,has_input,has_output,input_spec,output_spec,init,occ,map,score,failed,succ,type},%
  keywords=[2]{\#const,\#show,\#minimize,\#base,\#theory,\#count,\#external,\#program,\#script,\#end,\#heuristic,\#edge,\#project,\#show},%
  keywords=[3]{&,&dom,&sum,&diff,&show,&minimize},%
  morecomment=[l]{\#\ },%
  morecomment=[l]{\%\ },%
  commentstyle={\color{darkgray}}%
}
\lstset{numberblanklines=false,basicstyle=\ttfamily\small,language=clingo}
\lstset{escapeinside={*@}{@*}}

%----------------------------------------------------------------------------------------
%	TITLE PAGE
%----------------------------------------------------------------------------------------

\title[Likelihood]{Likelihood of Concern Satisfaction} % The short title appears at the bottom of every slide, the full title is only on the title page

\author{Thanh H. Nguyen} % Your name
\institute[NMSU] % Your institution as it will appear on the bottom of every slide, may be shorthand to save space
{
New Mexico State University \\ % Your institution for the title page
\medskip
\textit{tnguyen@cs.nmsu.edu} % Your email address
}
\date{\today} % Date, can be changed to a custom date

\begin{document}

\begin{frame}
\titlepage % Print the title page as the first slide
\end{frame}

% Slide 1:

%\begin{frame}
%\frametitle{Overview} % Table of contents slide, comment this block out to remove it
%\tableofcontents % Throughout your presentation, if you choose to use \section{} and \subsection{} commands, these will automatically be printed on this slide as an overview of your presentation
%\end{frame}

%----------------------------------------------------------------------------------------
%	PRESENTATION SLIDES
%----------------------------------------------------------------------------------------

%------------------------------------------------
\section{Content} 

\begin{frame}
	\frametitle{Physical CPS System}
	\begin{definition}
		\label{def:physical_CPS_system} 
		A physical CPS system $S$ is a tuple ($C, A, F, R$) where:
		%
		\begin{list}{$\bullet$}{\itemsep=0pt \parsep=1pt \topsep=1pt \leftmargin=12pt} 
			\item $C$ is a set of physical components.
			\item $A$ is a finite set of actions that can be execute over CPS system.
			\item $F$ is a finite set of fluent literals.
			\item $R$ is a set of relations that map each physical component $c \in C$ with a set of physical component properties that are defined in CPS Ontology.  \\
			For any $r \in R$, $r : C \longrightarrow 2^{P}$. $P$ is set of all properties that are defined in CPS ontology. The predicate {\tt relation(c,p) $\in R$} denotes that component $c \in C$ is related with property $p \in P$ in system $S$. 		 
		\end{list}
		%
	\end{definition}
\end{frame}

\begin{frame}[fragile]
	\frametitle{Idea: Likelihood of Concerns Satisfaction}
	\begin{itemize}
		\item A component was added to calculate the likelihood that concerns are satisfied. This was computed by:
		\begin{itemize}
			\item Percentage of properties/requirements with positive impact that are satisfied the concerns vs the total number of properties/requirements with positive impact, and
			\item Recursively aggregating the likelihood that its subconcerns are satisfied.
		\end{itemize}	
	\end{itemize}
\end{frame}

\begin{frame}[fragile]
	\frametitle{Represent the knowledge}
	\begin{itemize}
		\item {\bf Step 1:} Representation of concerns, properties and their relations from CPS Ontology by predicates {\tt concern/1}, {\tt property/1}, {\tt subconcern/2}, {\tt addressedBy/2}.  
		\item {\bf Step 2:} Representation the \emph{polarity} impacts of properties/requirements from CPS Ontology by predicates {\tt addressesPolarity(R,P)}. In which, {\tt P=pos/neg} denotes that properties/requirements $R$ impacts positively/negatively respectively. 
	\end{itemize}
\end{frame}
%------------------------------------------------

%----------------------------------------------------------------------------------------
%\bibliographystyle{aaai}
%\begin{thebibliography}{}
%\bibitem[\protect\citeauthoryear{Berners-Lee, Hendler, and
%                Lassila}{2001}]{lee01a}
%        Berners-Lee, T.; Hendler, J.; and Lassila, O. 2001. {The Semantics Web.} {\em Scientific American} 284(5):34--43.
        
%\end{thebibliography}
\end{document} 