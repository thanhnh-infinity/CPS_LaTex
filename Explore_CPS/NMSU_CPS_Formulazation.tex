%%%%%%%%%%%%%%%%%%%%%%%%%%%%%%%%%%%%%%%%%
% Beamer Presentation
% LaTeX Template
% Version 1.0 (10/11/12)
%
% This template has been downloaded from:
% http://www.LaTeXTemplates.com
%
% License:
% CC BY-NC-SA 3.0 (http://creativecommons.org/licenses/by-nc-sa/3.0/)
%
%%%%%%%%%%%%%%%%%%%%%%%%%%%%%%%%%%%%%%%%%

%----------------------------------------------------------------------------------------
%	PACKAGES AND THEMES
%----------------------------------------------------------------------------------------

\documentclass{beamer}

\mode<presentation> {

% The Beamer class comes with a number of default slide themes
% which change the colors and layouts of slides. Below this is a list
% of all the themes, uncomment each in turn to see what they look like.

%\usetheme{default}
%\usetheme{AnnArbor}
%\usetheme{Antibes}
%\usetheme{Bergen}
%\usetheme{Berkeley}
%\usetheme{Berlin}
%\usetheme{Boadilla}
%\usetheme{CambridgeUS}
%\usetheme{Copenhagen}
%\usetheme{Darmstadt}
%\usetheme{Dresden}
%\usetheme{Frankfurt}
%\usetheme{Goettingen}
%\usetheme{Hannover}
%\usetheme{Ilmenau}
%\usetheme{JuanLesPins}
%\usetheme{Luebeck}
\usetheme{Madrid}
%\usetheme{Malmoe}
%\usetheme{Marburg}
%\usetheme{Montpellier}
%\usetheme{PaloAlto}
%\usetheme{Pittsburgh}
%\usetheme{Rochester}
%\usetheme{Singapore}
%\usetheme{Szeged}
%\usetheme{Warsaw}

% As well as themes, the Beamer class has a number of color themes
% for any slide theme. Uncomment each of these in turn to see how it
% changes the colors of your current slide theme.

%\usecolortheme{albatross}
%\usecolortheme{beaver}
%\usecolortheme{beetle}
%\usecolortheme{crane}
%\usecolortheme{dolphin}
%\usecolortheme{dove}
%\usecolortheme{fly}
%\usecolortheme{lily}
%\usecolortheme{orchid}
%\usecolortheme{rose}
%\usecolortheme{seagull}
%\usecolortheme{seahorse}
%\usecolortheme{whale}
%\usecolortheme{wolverine}

%\setbeamertemplate{footline} % To remove the footer line in all slides uncomment this line
%\setbeamertemplate{footline}[page number] % To replace the footer line in all slides with a simple slide count uncomment this line

%\setbeamertemplate{navigation symbols}{} % To remove the navigation symbols from the bottom of all slides uncomment this line
}

\usepackage{graphicx} % Allows including images
\usepackage{booktabs} % Allows the use of \toprule, \midrule and \bottomrule in tables
\usepackage{algorithm}
\usepackage{algpseudocode}

\usepackage{listings}
\lstdefinelanguage{clingo}{
  keywordstyle=[1]\usefont{OT1}{cmtt}{m}{n},%
  keywordstyle=[2]\textbf,%
  keywordstyle=[3]\usefont{OT1}{cmtt}{m}{n},%\textit
  alsoletter={\#,\&},%
  keywords=[1]{not,from,import,exists,if,else,return,while,break,and,or,for,in,del,and,class,subclass,operation,has_input,has_output,input_spec,output_spec,init,occ,map,score,failed,succ,type},%
  keywords=[2]{\#const,\#show,\#minimize,\#base,\#theory,\#count,\#external,\#program,\#script,\#end,\#heuristic,\#edge,\#project,\#show},%
  keywords=[3]{&,&dom,&sum,&diff,&show,&minimize},%
  morecomment=[l]{\#\ },%
  morecomment=[l]{\%\ },%
  commentstyle={\color{darkgray}}%
}
\lstset{numberblanklines=false,basicstyle=\ttfamily\small,language=clingo}
\lstset{escapeinside={*@}{@*}}

%----------------------------------------------------------------------------------------
%	TITLE PAGE
%----------------------------------------------------------------------------------------

\title[Truthworthiness value]{Exploring current issues in CPS Theory ($\mathcal{S},I$)} % The short title appears at the bottom of every slide, the full title is only on the title page

\author{Thanh H. Nguyen} % Your name
\institute[NMSU] % Your institution as it will appear on the bottom of every slide, may be shorthand to save space
{
New Mexico State University \\ % Your institution for the title page
\medskip
\textit{tnguyen@cs.nmsu.edu} % Your email address
}
\date{\today} % Date, can be changed to a custom date

\begin{document}

\begin{frame}
\titlepage % Print the title page as the first slide
\end{frame}

% Slide 1:

\begin{frame}
\frametitle{Overview} % Table of contents slide, comment this block out to remove it
\tableofcontents % Throughout your presentation, if you choose to use \section{} and \subsection{} commands, these will automatically be printed on this slide as an overview of your presentation
\end{frame}

%----------------------------------------------------------------------------------------
%	PRESENTATION SLIDES
%----------------------------------------------------------------------------------------

%------------------------------------------------
\section{Issue 1: Complicated Relations between Properties and Concerns} 
\begin{frame}
	\frametitle{Issue 1: Problem Description}
	\begin{itemize}
		\item In CPS theory, there are 2 most important relations between $concern$ and $property$: $addBy(C,P)$ and $subconcern(C,C_1)$
		\item A concern C is satisfied iff {\bf all sub-concerns of C} are satisfied AND {\bf all properties that address for C} are satisfied.
		\item However, there exists a case that: $\exists p_1,p_2 \in P, c \in C$ and $addBy(c,p_1) \land addBy(c,p_2)$ and $c$ has no sub-concerns, $sat(c)$ holds if $sat(p_1) \lor sat(p_2)$ holds.
		\item Example, properties \{$two\_factos\_auth, finger\_printing\_auth$\} address concern $Authorization$.  concern $Authorization$ is satisfied if a physical device uses $two\_factos\_auth$ OR $finger\_printing\_auth$. 
		\item This idea is not appropriate with current CPS Theory.
	\end{itemize}
\end{frame}

\begin{frame}[fragile]
	\frametitle{Issue 1: Solution -- Extend CPS Ontology}
	\begin{itemize}
		\item We propose new terminology: $Supplementary$ $Property$ (SP)
		\item We propose new relation between $Supplementary$ $Property$ and $Property$: $supportFor(SP,P)$ denotes that supplementary property $SP$ supports for property $P$.
		\item A property $P$ is satisfied IFF the truth value of $P$ is $true$ OR one of supplementary properties of $P$ is True. \\
		{\tt holds(sat(P),S) :- holds(sat(SP),S), supportFor(SP,P).}
		\item In CPS theory ($\mathcal{S},I$). The relation $r \in R$ denotes the relation between a component $c$ and a set of supplementary properties $sp$. The predicate $relation(c,sp)$ denotes that component $c$ is related with supplementary property $sp$.  
	\end{itemize}
\end{frame}

\begin{frame}[fragile]
	\frametitle{Issue 1: Changes in Planning Engine}
	\begin{itemize}
		\item The extension of CPS Ontology will support to improve the reasoning and the mitigation strategies generation.
		\item The CPS action is not only able to turn ON/OFF the supplementary property (make the truth values of these properties True or False), but also is able to switch component to use authentication function between \{$two\_factors\_auth$ and $finger\_printing\_auth$\}.
		\item Generate the more powerful mitigation strategies (multiple types of actions which changes truth value of supplementary property AND changes the relation between component and supplementary properties). 	   
	\end{itemize}
\end{frame}

%------------------------------------------------
\section{Issue 2: The Conflicts between Properties} 
\begin{frame}
	\frametitle{Issue 2: Problem Description}
	\begin{itemize}
		\item We 	   
	\end{itemize}
\end{frame}

\begin{frame}[fragile]
	\frametitle{Issue 2: Solution -- Likelihood of Concern Satisfaction}
	\begin{itemize}
		\item We 	   
	\end{itemize}
\end{frame}


%------------------------------------------------
\section{Issue 3: Timing Constraits} 
\begin{frame}
	\frametitle{Issue 3: Problem Description}
	\begin{itemize}
		\item We 	   
	\end{itemize}
\end{frame}

\begin{frame}[fragile]
	\frametitle{Issue 3: Solution}
	\begin{itemize}
		\item We 	   
	\end{itemize}
\end{frame}

%------------------------------------------------
\section{Issue 4: More Advanced Trustworthiness Queries} 
\begin{frame}
	\frametitle{More Advanced Trustworthiness Queries}
	\begin{itemize}
		\item We 	   
	\end{itemize}
\end{frame}


%----------------------------------------------------------------------------------------
%\bibliographystyle{aaai}

%\begin{thebibliography}{}

%\bibitem[\protect\citeauthoryear{Berners-Lee, Hendler, and
%                Lassila}{2001}]{lee01a}
%        Berners-Lee, T.; Hendler, J.; and Lassila, O. 2001. {The Semantics Web.} {\em Scientific American} 284(5):34--43.
        
%\end{thebibliography}
\end{document} 